%%%%%%%%%%%%%%%%%%%%%%%%%%%%%%%%%%%%%%%%%%%%%%%%%%%%%%%%%%%%%%%%%%%%%%%%%%%%
%%
%W  pargap.tex		NQL Doc				René Hartung
%%
%H  $Id: pargap.tex,v 1.3 2010/03/31 13:42:54 gap Exp $
%%

%%%%%%%%%%%%%%%%%%%%%%%%%%%%%%%%%%%%%%%%%%%%%%%%%%%%%%%%%%%%%%%%%%%%%%%%%%%%
\Chapter{On a parallel nilpotent quotient algorithm}

We included a parallel version of \NQL's nilpotent quotient
algorithm using the \ParGap-package of \GAP; see~\cite{ParGap}. In
this chapter, we outline the basic usage of this parallel part of the
\NQL-package. For further details on the parallel \GAP-sessions we refer to the
\ParGap-manual~\cite{ParGap}. We note that the \ParGap-package has some
bottlenecks in practice. Nevertheless the significant speed-up of our
computations on a multiple-core system shows that this is a reasonable
extension of the \NQL-package.\medskip

For using the parallel version of the nilpotent quotient algorithm,
you will need to install the \ParGap-package as described in its
manual~\cite{ParGap}. When using Version~1.1.2 of the \ParGap-package,
you will need to apply the following patch to `pargap/lib/masslave.g'
as otherwise the \ParGap-session may crash. On a linux machine you can
simply use `patch \< ../../nql/gap/pargap/patch' from within the directory
`pargap/lib/'.

\beginexample
--- masslave.g	2001-11-16 13:17:04.000000000 +0100
+++ masslave.g	2009-05-06 12:20:19.000000000 +0200
@@ -467,8 +467,9 @@
   if Length(deltas)>1 then max2 := Maximum(max2, deltas[Length(deltas)-1]); fi;
   max1 := deltas[Length(deltas)];
   pos1 := Position( List(slaveArray, x->realtime-x.time), max1 );
-  if max1 > slaveTaskTimeFactor and max1 > 30
-     and slaveTaskTime[pos2].total > 60 then
+  if max1 > slaveTaskTimeFactor and
+     max1 > 30 and pos2 <> fail and 
+     slaveTaskTime[pos2].total > 60 then
     Print("SLAVE ",pos1," SEEMS DEAD!!\n");
   fi;
 end);
\endexample

Now, you are ready for creating a \ParGap-session and you can load
the \NQL-package from within \ParGap\ using `RequirePackage' as usual.
The same methods as described previously are available. The following
example shows the application of the `NilpotentQuotient'-method to
the Grigorchuk group on a quad-core machine. Note that the significant
speed-up of the nilpotent quotient algorithm is especially noticeable
for large nilpotent quotients. This parallel version of \NQL\ successfully
computes some nilpotent quotients which normally took more than a month
to complete.

\beginexample
GAP4, Version: 4.4.12 of 17-Dec-2008, i686-pc-linux-gnu-gcc
GAP4, Version: 4.4.12 of 17-Dec-2008, i686-pc-linux-gnu-gcc
GAP4, Version: 4.4.12 of 17-Dec-2008, i686-pc-linux-gnu-gcc
GAP4, Version: 4.4.12 of 17-Dec-2008, i686-pc-linux-gnu-gcc
GAP4, Version: 4.4.12 of 17-Dec-2008, i686-pc-linux-gnu-gcc
gap> TOPCnumSlaves;
4
gap> RequirePackage("NQL");
true
gap> G:=ExamplesOfLPresentations(1);
<L-presented group on the generators [ a, b, c, d ]>
gap> SetInfoLevel(InfoNQL,1);
gap> NilpotentQuotient(G,2);
#I  Class 1: 3 generators with relative orders: [ 2, 2, 2 ]
#I  Computing a polycyclic presentation for the covering group...
#I  Checking the consistency relations...
master -> 1: (AGGLOM_TASK): [ [ -3, 1 ], [ -3, 2 ], [ -2, 1 ], [ 2, -1 ],
  [ 3, -1 ] ]
master -> 2: (AGGLOM_TASK): [ [ 3, -2 ], [ 1 ], [ 2 ], [ 3 ] ]
1 -> master: [ [ 0, 0, 0, 0, 0, -2, 0 ], [ 0, 0, 0, 0, 0, 0, -2 ],
  [ 0, 0, 0, 0, -2, 0, 0 ], [ 0, 0, 0, 0, -2, 0, 0 ],
  [ 0, 0, 0, 0, 0, -2, 0 ] ]
2 -> master: [ [ 0, 0, 0, 0, 0, 0, -2 ], [ 0, 0, 0, 0, 0, 0, 0 ],
  [ 0, 0, 0, 0, 0, 0, 0 ], [ 0, 0, 0, 0, 0, 0, 0 ] ]
#I  Broadcasting the slaves...
#I  Inducing the endomorphisms...
master -> 1:  1
master -> 2:  2
master -> 3:  3
master -> 4:  4
3 -> master: [ 2, 1 ]
UPDATE: [ 3, [ 2, 1 ] ]
1 -> master: [ 2, 1, 8, 1 ]
UPDATE: [ 1, [ 2, 1, 8, 1 ] ]
2 -> master: [ 2, 1, 3, 1, 4, 1 ]
UPDATE: [ 2, [ 2, 1, 3, 1, 4, 1 ] ]
master -> 1:  5
master -> 2:  6
master -> 3:  7
4 -> master: [ 4, -1, 6, -1, 10, -1 ]
UPDATE: [ 4, [ 4, -1, 6, -1, 10, -1 ] ]
1 -> master: [ 6, 1, 8, 2 ]
UPDATE: [ 5, [ 6, 1, 8, 2 ] ]
2 -> master: [ 4, 2, 6, 1, 7, 1, 10, 1 ]
UPDATE: [ 6, [ 4, 2, 6, 1, 7, 1, 10, 1 ] ]
3 -> master: [ 6, 1 ]
UPDATE: [ 7, [ 6, 1 ] ]
master -> 1:  8
master -> 2:  9
master -> 3:  10
1 -> master: [ 10, 1 ]
UPDATE: [ 8, [ 10, 1 ] ]
2 -> master: [  ]
UPDATE: [ 9, [  ] ]
3 -> master: [ 10, -1 ]
UPDATE: [ 10, [ 10, -1 ] ]
#I  Broadcasting the slaves...
#I  Mapping the relations...
master -> 1:  1
master -> 2:  2
master -> 3:  3
master -> 4:  4
1 -> master: [ 0, 0, 0, 0, 1, 0, 0, 0, 0, 0 ]
2 -> master: [ 0, 0, 0, 2, 0, 1, 1, 0, 0, 1 ]
3 -> master: [ 0, 0, 0, 0, 0, 1, 0, 0, 0, 0 ]
4 -> master: [ 0, 0, 0, 0, 0, 0, 1, 0, 0, 0 ]
master -> 1:  5
master -> 2:  6
master -> 3:  7
1 -> master: [ 0, 0, 0, 1, 0, 1, 1, 0, 0, 1 ]
2 -> master: [ 0, 0, 0, 0, 0, 0, 0, 0, 2, 0 ]
3 -> master: [ 0, 0, 0, 0, 0, 0, 0, 4, 6, 4 ]
#I  Start spinning...
#I  Extend the quotient system...
#I  Class 2: 2 generators with relative orders: [ 2, 2 ]
Pcp-group with orders [ 2, 2, 2, 2, 2 ]
\endexample

Note that the only difference in the parallel version of the \NQL-package
is a parallel version of the operation `ExtendQuotientSystem'. This latter
operation covers the induction step of the nilpotent quotient algorithm.

%%%%%%%%%%%%%%%%%%%%%%%%%%%%%%%%%%%%%%%%%%%%%%%%%%%%%%%%%%%%%%%%%%%%%%%%%%%%
%%
%E  pargap.tex .  .  .  .  .  .  .  .  .  .  .  .  .  .  .  .  .  ends here
%%
